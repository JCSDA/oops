% (C) Copyright 2009-2016 ECMWF.
% 
% This software is licensed under the terms of the Apache Licence Version 2.0
% which can be obtained at http://www.apache.org/licenses/LICENSE-2.0. 
% In applying this licence, ECMWF does not waive the privileges and immunities 
% granted to it by virtue of its status as an intergovernmental organisation nor
% does it submit to any jurisdiction.

\documentclass[12pt]{article}
\parskip=12pt
\parindent=0pt
\title{A Two Level Quasi-geostrophic Model}
\author{Mike Fisher, ECMWF}

\begin{document}
\maketitle

\section{Introduction}
This note describes a simple two-level quasi-gestrophic model, intended
for use as a ``toy'' system with which to conduct idealised studies of
data assimilation methods. In developing the model, the emphasis has been
placed on speed and convenience rather than accuracy and conservation.

\section{The Continuous Equations}
The equations of the two-level model are given by Fandry and Leslie (1984)
(see also Pedlosky, 1979 pp386-393), and are expressed in terms of
non-dimensionalised variables:
\begin{equation}
  \frac{{\rm D}q_1}{{\rm D}t} = \frac{{\rm D}q_2}{{\rm D}t} = 0
\end{equation}
where $q_1$ and $q_2$ denote the quasi-geostrophic potential vorticity
on each of the two layers, with a subscript 1 denoting the upper layer:
\begin{eqnarray}
q_1 &=& \nabla^2 \psi_1 - F_1 (\psi_1 -\psi_2 ) + \beta y \label{q_1}\\
q_2 &=& \nabla^2 \psi_2 - F_2 (\psi_2 -\psi_1 ) + \beta y + R_s \label{q_2}
\end{eqnarray}

Here, $\beta$ is the (non-dimensionalised) northward derivative of the
Coriolis parameter, and $R_s$ represents orography or heating.

The model domain is assumed to be cyclic in the zonal direction, and
the meridional velocity is assumed to vanish one grid space to the
north and south of the domain.

\section{Details of the Non-Dimensionalisation}
The non-dimensionalisation is standard, but is given here for completeness.
We define a typical length scale $L$, a typical velocity $U$, the depths
of the upper and lower layers $D_1$ and $D_2$, the Coriolis parameter
at the southern boundary $f_0$ and its northward derivative $\beta_0$,
the acceleration due to gravity $g$, the difference in potential
temperature across the layer interface $\Delta\theta$, and the mean
potential temperature $\overline\theta$.

Denoting dimensional time, spatial coordinates and velocities with
tildes, we have:
\begin{eqnarray*}
t &=& \tilde t \frac{\overline U}{L} \\
x &=& \frac{\tilde x}{L} \\
y &=& \frac{\tilde y}{L} \\
u &=& \frac{\tilde u}{\overline U} \\
v &=& \frac{\tilde v}{\overline U} \\
F_1 &=& \frac{f_0^2 L^2}{D_1 g \Delta\theta / {\overline\theta}} \\
F_2 &=& \frac{f_0^2 L^2}{D_2 g \Delta\theta / {\overline\theta}} \\
\beta &=& \beta_0 \frac{L^2}{\overline U}
\end{eqnarray*}

The Rossby number is $\epsilon = {\overline U} / f_0 L$.

\section{Solution Algorithm}
The prognostic variable of the model is streamfunction, defined on a
rectangular grid of dimension {\tt nx} $\times$ {\tt ny}.
The grid indices increase in the eastward and northward directions.

The time-stepping algorithm is designed for speed rather than accuracy,
and is accurate only to first-order in $\Delta t$. It has the practical
advantage that a timestep may be performed given information at only a
single time-level.

In principle, a timestep could start from values of streamfunction at a
single time, $t$, and return values of streamfunction at $t+\Delta t$.
However, to make wind (and potential vorticity) available to the
analysis layer (e.g. to allow assimilation of wind observations), it is
more convenient to split the timestep as follows:

\subsection{Before an integration of the model}

Before an integration of the model, values of wind and potential
vorticity are calculated by {\tt c\_qg\_prepare\_integration}.

The velocity at each gridpoint is calculated using centred,
finite-difference approximations to:

\begin{equation}
u = -\frac{\partial \psi}{\partial y} ,\qquad
v =  \frac{\partial \psi}{\partial x} .
\end{equation}

Values of $\psi$ one grid-space to the north and south of the grid are
required in order to calculate the $u$-component of velocity on the
first and last grid row. These values are user-supplied constants,
and determine the mean zonal velocity in each layer, which remains
constant throughout the integration. (Note that the condition that
$v$ should vanish at the northern and southern boundaries implies
that $\psi$ is independent of $x$ at the boundaries.)

Potential vorticity is calculated using equations
\ref{q_1} and \ref{q_2}. A standard 5-point finite-difference
approximation to the Laplacian operator is used.

\subsection{Steps evaluated at every timestep}

The following steps are repeated for each timestep:

\begin{enumerate}
\item For each gridpoint, $(x_{ij} ,y_{ij})$, the departure point is
calculated as:
\begin{equation}
  x^D_{ij} = x_{ij} - \frac{\Delta t}{\Delta x} u^t_{ij} ,\qquad
  y^D_{ij} = y_{ij} - \frac{\Delta t}{\Delta y} v^t_{ij} .
\end{equation}

\item The potential vorticity field at the end of the timestep is
calculated by interpolating to the departure point:
\begin{equation}
  q^{t+\Delta t}_{ij} = q(x^D_{ij}, y^D_{ij})
\end{equation}
The interpolation is bi-cubic. Advection from outside the domain is
handled by assuming the potential vorticity to be constant for all
points one grid-space or more outside the domain.
The boundary values of potential vorticity are supplied by the user.

\item The streamfunction corresponding to $q^{t+\Delta t}$ is determined
by inverting equations \ref{q_1} and \ref{q_2}, as described below. 

\item The velocity components at time $t+\Delta t$ are calculated from
the streamfunction.
\end{enumerate}

\subsection{Inversion of Potential Vorticity}
Applying $\nabla^2$ to equation \ref{q_1} and subtracting $F_1$ times
equation \ref{q_1} and $F_2$ times equation \ref{q_2} eliminates
$\psi_1$, and yields the following equation for $\psi_1$:
\begin{equation}
\nabla^2 q_1 -F_2 q_1 -F_1 q_2 = \nabla^2 \left( \nabla^2 \psi_1 \right)
                               - \left( F_1 + F_2 \right) \nabla^2 \psi_1
\label{2d_helmholz_eqn}
\end{equation}
This is a two-dimensional Helmholz equation, which can be solved 
for $\nabla^2 \psi_1$. The Laplacian can then be inverted to
determine $psi_1$. Once $\psi_1$ and $\nabla^2 \psi_1$ are known, the
streamfunction on level 2 can be determined by substitution into
equation \ref{q_1}.

Solution of the Helmholz equation and inversion of the Laplacian are achieved
using an FFT-based method. Applying a Fourier transform in the east-west
direction to equation \ref{2d_helmholz_eqn} gives a set of independent
equations for each wavenumber. In the case of the five-point discrete
Laplacian, these are tri-diagaonal matrix equations, which can be solved
using the standard (Thomas) algorithm.

\section{References}
Fandry, C.B. and L.M. Leslie, 1984: A Two-Layer Quasi-Geostrophic Model of
Summer Trough Formation in the Australian Subtropical Easterlies.
J.A.S., {\bf 41}, pp807-817.

Pedlosky, J., 1979: Geophysical Fluid Dynamics. {\it Springer-Verlag}.
\end{document}
